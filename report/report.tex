\documentclass[hidelinks]{article}
%
%%%%%%%%%%%%%%%%%%%%%%%%%%%%%%%%%%%%%%%%%%%%%%%%%%%%%%%%%%%%%%%
% START CUSTOM INCLUDES & DEFINITIONS
%%%%%%%%%%%%%%%%%%%%%%%%%%%%%%%%%%%%%%%%%%%%%%%%%%%%%%%%%%%%%%%
%
\usepackage{amsmath}
\usepackage{parskip} %noident everywhere
\usepackage{hyperref} % Show hyperlinks - claudio
\hypersetup{
    colorlinks = true
    linkcolor = blue
    urlcolor = red
    }
%
%%%%%%%%%%%%%%%%%%%%%%%%%%%%%%%%%%%%%%%%%%%%%%%%%%%%%%%%%%%%%%%
% END CUSTOM INCLUDES & DEFINITIONS
%%%%%%%%%%%%%%%%%%%%%%%%%%%%%%%%%%%%%%%%%%%%%%%%%%%%%%%%%%%%%%%
%
\pdfobjcompresslevel=0
%
\title{Submarine Mission Report}
\author{Claudio Vestini}
\date{October 2024}
\begin{document}
\maketitle
%
\section{Motivation and Preliminary Steps}
This brief report will concern the B1 submarine coding practical.
\newline
We were tasked with designing the controller to guide a Submarine through its cave mission by tracking a given reference to avoid collisions with the cave boundaries.
\newline
Initial steps taken before starting development:
%
\begin{enumerate}
    \item Create and activate a virtual environment with the given requirements (numpy, matplotlib and pandas packages)
    \item Fork project repository to my GitHub account
    \item Set up a .env file to add local packages onto Python PATH
    \item Make sure running files do not give any errors before branching off `main'
\end{enumerate}
%
\section{Mission Data Extraction}
The first task was to obtain mission data from the given .csv file.
\newline
I started by creating a new branch to modify the Mission class within. I then implemented a new classmethod to extract each column of the mission.csv file into a separate variable, and then return the data as an instance of the Mission class.
\newline
The next step was to test the new functionality by using the Trajectory class's plotting methods. Once I made sure the new method was working correctly, I merged the branch back into `main', and deleted the unused branch.
%
\section{Controller Implementation}
To implement the PD controller, I first analysed the given code to infer the system's dynamics. The submarine progresses at constant speed in the $x$ direction, so needs only be controlled in the y direction (another hint to this is that we only have one set of reference values).
\newline
By inspection of the transition method, given drag $D$, velocity $V_y$, actuator input $u$ and disturbance $d$, the force in the vertical direction is given by:
%
\begin{equation}
    F_y = - D \cdot V_y + K_{actuator} \cdot (u + d) \label{eq:force}
\end{equation}
%
Combining \eqref{eq:force} with Newton's second law we obtain acceleration:
%
\begin{equation}
    \frac{d^2y}{dt^2} = \frac{- D \cdot V_y}{m} + \frac{K_{actuator} \cdot (u + d)}{m} \label{eq:force}
\end{equation}
%
It is then very simple to obtain all matrices for the state space dynamics in canonic form:
\begin{equation}
    \begin{aligned}
        \dot{x}(t) &= A x(t) + B u(t) \\
        y(t) &= C x(t) + D u(t)
    \end{aligned}
    \label{eq:statespace}
\end{equation}
%
as:
%
\begin{equation}
    \displaystyle
    A = \begin{bmatrix}
            0 & 1 \\
            0 & -\frac{D}{m}
        \end{bmatrix}; \quad
    B = \begin{bmatrix}
            0 \\
            \frac{K_{actuator}}{m}
        \end{bmatrix}; \quad
    C = \begin{bmatrix}
            1 \\
            0
          \end{bmatrix}; \quad
    D = \begin{bmatrix}
            0
        \end{bmatrix}
\end{equation}
%
The PD controller is then simply implemented
%
\end{document}